%%%%%%%%%%%%%%%%%%%%%%%%%%%%%%%%%%%%%%%%%
% Thin Sectioned Essay
% LaTeX Template
% Version 1.0 (3/8/13)
%
% This template has been downloaded from:
% http://www.LaTeXTemplates.com
%
% Original Author:
% Nicolas Diaz (nsdiaz@uc.cl) with extensive modifications by:
% Vel (vel@latextemplates.com)
%
% License:
% CC BY-NC-SA 3.0 (http://creativecommons.org/licenses/by-nc-sa/3.0/)
%
%%%%%%%%%%%%%%%%%%%%%%%%%%%%%%%%%%%%%%%%%

%----------------------------------------------------------------------------------------
%	PACKAGES AND OTHER DOCUMENT CONFIGURATIONS
%----------------------------------------------------------------------------------------

\documentclass[a4paper, 11pt]{article} % Font size (can be 10pt, 11pt or 12pt) and paper size (remove a4paper for US letter paper)

\usepackage[protrusion=true,expansion=true]{microtype} % Better typography
\usepackage{graphicx} % Required for including pictures
\usepackage{wrapfig} % Allows in-line images

\usepackage{mathpazo} % Use the Palatino font
\usepackage[T1]{fontenc} % Required for accented characters

% latex packages for source code examples
\usepackage{listings}
\usepackage{cite}
\usepackage{caption}
\usepackage{upquote}
\usepackage{xcolor}
\usepackage{float}

\DeclareCaptionFont{white}{\color{white}}
\DeclareCaptionFormat{listing}{\colorbox{gray}{\parbox{\textwidth}{#1#2#3}}}
\captionsetup[lstlisting]{format=listing,labelfont=white,textfont=white}

\lstset{
    keywordstyle=\bfseries\ttfamily\color[rgb]{0,0,1},
    identifierstyle=\ttfamily,
    commentstyle=\color[rgb]{0.133,0.545,0.133},
    stringstyle=\ttfamily\color[rgb]{0.627,0.126,0.941},
    showstringspaces=false,
    basicstyle=\small,
    numberstyle=\footnotesize,
    numbers=left,
    stepnumber=1,
    numbersep=10pt,
    tabsize=2,
    breaklines=true,
    prebreak = \raisebox{0ex}[0ex][0ex]{\ensuremath{\hookleftarrow}},
    breakatwhitespace=false,
    aboveskip={1.5\baselineskip},
    columns=fixed,
    upquote=true,
    extendedchars=true,
    % frame=bottomline,
    inputencoding=utf8,
    showspaces=false
}

\lstdefinestyle{cli}{
    linewidth=\textwidth,
    xleftmargin=0.01\textwidth,
    xrightmargin=0.01\textwidth,
    numbers=none
}

\linespread{1.05} % Change line spacing here, Palatino benefits from a slight increase by default

\makeatletter
\renewcommand\@biblabel[1]{\textbf{#1.}} % Change the square brackets for each bibliography item from '[1]' to '1.'
\renewcommand{\@listI}{\itemsep=0pt} % Reduce the space between items in the itemize and enumerate environments and the bibliography

\renewcommand{\maketitle}{ % Customize the title - do not edit title and author name here, see the TITLE block below
\begin{flushright} % Right align
{\LARGE\@title} % Increase the font size of the title

\vspace{50pt} % Some vertical space between the title and author name

{\large\@author} % Author name
\\\@date % Date

\vspace{40pt} % Some vertical space between the author block and abstract
\end{flushright}
}

%----------------------------------------------------------------------------------------
%	TITLE
%----------------------------------------------------------------------------------------

\title{\textbf{LFC log analyzer}\\ % Title
LCG.CESNET.cz site} % Subtitle

\author{\textsc{Radek Ludacka, Jiri Chudoba, Martin Adam (English corection)} % Author
\\{\textit{Academy of Sciences of the Czech Republic - Institute of Physics}}} % Institution

\date{\today} % Date

%----------------------------------------------------------------------------------------

\begin{document}

\maketitle % Print the title section

%----------------------------------------------------------------------------------------
%	ABSTRACT AND KEYWORDS
%----------------------------------------------------------------------------------------

%\renewcommand{\abstractname}{Summary} % Uncomment to change the name of the abstract to something else

\begin{abstract}
LFC (LCG file catalog) is standard grid catalog intended to interconnect a logical file name (LFN) that is the same for the whole grid with a physical file location (where is file really stored). The reason why the LFC log analyzer was developed is: LFC log files have to be analyzed to determine LFC server utilization and find possible bottlenecks of LFC usage. The LFC log analyzer should provide usage statistics that enable presenting selected items or items with particular value. These statistics can be used to analyze LFC usage and improve its efficiency.
\end{abstract}

\hspace*{3,6mm}\textit{Keywords:} LFC, grid, statistics, analyzer, log file, logging % Keywords

\vspace{30pt} % Some vertical space between the abstract and first section

%----------------------------------------------------------------------------------------
%	ESSAY BODY
%----------------------------------------------------------------------------------------

\section*{Lfc log file description}

Standard LFC log file path is /var/log/lfc/. This directory contains log file names log. File log contains current log items. The directory with log files contains also other (compressed) log files named log-YYYYMMDD.gz where YYYY is year, MM mouth and DD day of particular log date. These files contain log items written down in previous days.

\newpage

\section{Lfc log row and item}

Each row in the log file contains log time, process id, thread id and a function name that defines what happens. Figure \ref{log:row} shows an example of LFC log file row.

\begin{lstlisting}[label={log:row},caption={Cns\_srv\_startsess row example}, breaklines]
11/03 04:28:41.596  9721,5 Cns_srv_readdir: NS092 - readdir request by /DC=es/DC=irisgrid/O=ugr/CN=Julio.Lozano.Bahilo (432,117,104) from polgrid115.in2p3.fr
\end{lstlisting}

In the beginning of each row is date and time that describes when the function was logged. Date is in MM/DD format. Time is in hour:minutes:seconds.miliseconds format. Process id and thread id split by a comma are written down after date and time. These numbers define process and thread that wrote the row of the log file. Function name is placed after thread id. This was the general part of each log row.

Each log row is part of a log item. The log item is composed from several rows (usually 3 or 4). Each item row has the same process id, thread id and function name. Figure~\ref{log:item} shows an example of Cns\_srv\_startsess item. Threads share one log file therefore items and item rows are interleaved each other.


\begin{lstlisting}[label={log:item},caption={Cns\_srv\_startsess}]
11/03 04:28:42.014  9721,1 Cns_srv_startsess: NS092 - startsess request by /DC=es/DC=irisgrid/O=ugr/CN=Julio.Lozano.Bahilo (432,117,104) from polgrid90.in2p3.fr
11/03 04:28:42.015  9721,1 Cns_srv_startsess: NS098 - startsess ()
11/03 04:28:42.015  9721,1 Cns_srv_startsess: returns 0
\end{lstlisting}

\newpage

\section{How are lfc commands written down to log file}

Most of LFC commands are not mapped to only one item. Many LFC commands are written down as several items to one log file. This document will present several LFC commands and how they are written down in a log file. Most of LFC command examples are described in two variants: success and failed.

\subsection{lfc-ls}

lfc-ls - provides very similar functionality as standard UNIX ls command. lfc-ls command enables to list an arbitrary LFC file or directory. How the lfc-ls command is written down to a log file depends on lfc-ls destination type. If destination is a file then the only one lstat item is written down to the log file. If destination is a directory then lstat, opendir, readdir and closedir sequence is written down to the log file. Number of readdir items may vary depending on number of files and directories in the target directory. Command lfc-ls allows recursive mode, which is set by -R option. If recursive mode is used then only one lstat and sequence of opendir, readdir and closedir are written down to the log file.

\begin{lstlisting}[label={log:lfcls1},caption={lfc-ls /grid/voce/ludacka/text\_file2.txt}]
07/23 14:38:39.123 20713,0 Cns_srv_lstat: NS092 - lstat request by /DC=org/DC=terena/DC=tcs/C=CZ/O=Czech Technical University in Prague/CN=Radek Ludacka 364520 (613,101) from ui1.egee.cesnet.cz
07/23 14:38:39.123 20713,0 Cns_srv_lstat: NS098 - lstat 0 /grid/voce/ludacka/text_file2.txt
07/23 14:38:39.125 20713,0 Cns_srv_lstat: returns 0
\end{lstlisting}

\begin{lstlisting}[label={log:lfcls2},caption={lfc-ls /grid/voce/ludacka/text\_file2.txt}]
07/09 09:58:28.883 20713,0 Cns_srv_lstat: NS092 - lstat request by /DC=org/DC=terena/DC=tcs/C=CZ/O=Czech Technical University in Prague/CN=Radek Ludacka 364520 (613,101) from ui1.egee.cesnet.cz
07/09 09:58:28.883 20713,0 Cns_srv_lstat: NS098 - lstat 0 /
07/09 09:58:28.884 20713,0 Cns_srv_lstat: returns 0
07/09 09:58:28.953 20713,0 Cns_srv_opendir: NS092 - opendir request by /DC=org/DC=terena/DC=tcs/C=CZ/O=Czech Technical University in Prague/CN=Radek Ludacka 364520 (613,101) from ui1.egee.cesnet.cz
07/09 09:58:28.953 20713,0 Cns_srv_opendir: NS098 - opendir / 
07/09 09:58:28.954 20713,0 Cns_srv_opendir: returns 0
07/09 09:58:28.985 20713,0 Cns_srv_readdir: NS092 - readdir request by /DC=org/DC=terena/DC=tcs/C=CZ/O=Czech Technical University in Prague/CN=Radek Ludacka 364520 (613,101) from ui1.egee.cesnet.cz
07/09 09:58:28.994 20713,0 Cns_srv_readdir: returns 0
07/09 09:58:29.033 20713,0 Cns_srv_readdir: NS092 - closedir request by /DC=org/DC=terena/DC=tcs/C=CZ/O=Czech Technical University in Prague/CN=Radek Ludacka 364520 (613,101) from ui1.egee.cesnet.cz
07/09 09:58:29.034 20713,0 Cns_srv_readdir: returns 0
\end{lstlisting}

\begin{lstlisting}[label={log:lfcls3},caption={lfc-ls /grid/voce/neexist - query to file that does not exists}]
07/12 11:22:40.421 20713,4 Cns_srv_lstat: NS092 - lstat request by /DC=org/DC=terena/DC=tcs/C=CZ/O=Czech Technical University in Prague/CN=Radek Ludacka 364520 (613,101) from ui1.egee.cesnet.cz
07/12 11:22:40.421 20713,4 Cns_srv_lstat: NS098 - lstat 0 /grid/voce/neexist
07/12 11:22:40.423 20713,4 Cns_srv_lstat: returns 2
\end{lstlisting}

\subsection{lfc-cr}

lfc-cr - uploads a file to a SE. This command also registers the file to LFC. 

\begin{lstlisting}[label={log:lcgcr1},caption={lcg-cr --vo voce -d dpm1.egee.cesnet.cz -l lfn:/grid/voce/ludacka/text\_file2.txt "file://\$PWD/text\_file2.txt"}]
07/10 14:25:22.109 20713,1 Cns_srv_lstat: NS092 - lstat request by /DC=org/DC=terena/DC=tcs/C=CZ/O=Czech Technical University in Prague/CN=Radek Ludacka 364520 (613,101) from ui1.egee.cesnet.cz
07/10 14:25:22.109 20713,1 Cns_srv_lstat: NS098 - lstat 0 /grid/voce/ludacka/text_file2.txt
07/10 14:25:22.111 20713,1 Cns_srv_lstat: returns 2
07/10 14:25:29.121 20713,3 Cns_srv_starttrans: NS092 - starttrans request by /DC=org/DC=terena/DC=tcs/C=CZ/O=Czech Technical University in Prague/CN=Radek Ludacka 364520 (613,101) from ui1.egee.cesnet.cz
07/10 14:25:29.121 20713,3 Cns_srv_starttrans: NS098 - starttrans (1.11.16-3)
07/10 14:25:29.121 20713,3 Cns_srv_starttrans: returns 0
07/10 14:25:29.121 20713,3 Cns_srv_statg: NS092 - statg request by /DC=org/DC=terena/DC=tcs/C=CZ/O=Czech Technical University in Prague/CN=Radek Ludacka 364520 (613,101) from ui1.egee.cesnet.cz
07/10 14:25:29.121 20713,3 Cns_srv_statg: NS098 - statg /grid/voce/ludacka/text_file2.txt 8da67819-67e1-494a-814f-ee9d8504f2f0
07/10 14:25:29.123 20713,3 Cns_srv_statg: returns 2
07/10 14:25:29.124 20713,3 Cns_srv_statg: NS092 - statg request by /DC=org/DC=terena/DC=tcs/C=CZ/O=Czech Technical University in Prague/CN=Radek Ludacka 364520 (613,101) from ui1.egee.cesnet.cz
07/10 14:25:29.124 20713,3 Cns_srv_statg: NS098 - statg /grid/voce/ludacka 
07/10 14:25:29.125 20713,3 Cns_srv_statg: returns 0
07/10 14:25:29.162 20713,3 Cns_srv_creat: NS092 - creat request by /DC=org/DC=terena/DC=tcs/C=CZ/O=Czech Technical University in Prague/CN=Radek Ludacka 364520 (613,101) from ui1.egee.cesnet.cz
07/10 14:25:29.162 20713,3 Cns_srv_creat: NS098 - creat /grid/voce/ludacka/text_file2.txt 8da67819-67e1-494a-814f-ee9d8504f2f0 664 22
07/10 14:25:29.173 20713,3 Cns_srv_creat: file 37393414 created
07/10 14:25:29.173 20713,3 Cns_srv_creat: returns 0
07/10 14:25:29.210 20713,3 Cns_srv_setfsizeg: NS092 - setfsizeg request by /DC=org/DC=terena/DC=tcs/C=CZ/O=Czech Technical University in Prague/CN=Radek Ludacka 364520 (613,101) from ui1.egee.cesnet.cz
07/10 14:25:29.210 20713,3 Cns_srv_setfsizeg: NS098 - setfsizeg 8da67819-67e1-494a-814f-ee9d8504f2f0 19
07/10 14:25:29.213 20713,3 Cns_srv_setfsizeg: returns 0
07/10 14:25:29.213 20713,3 Cns_srv_addreplica: NS092 - addreplica request by /DC=org/DC=terena/DC=tcs/C=CZ/O=Czech Technical University in Prague/CN=Radek Ludacka 364520 (613,101) from ui1.egee.cesnet.cz
07/10 14:25:29.213 20713,3 Cns_srv_addreplica: NS098 - addreplica 8da67819-67e1-494a-814f-ee9d8504f2f0 dpm1.egee.cesnet.cz srm://dpm1.egee.cesnet.cz/dpm/cesnet.cz/home/voce/generated/2013-07-10/file2a59eb51-b0ca-4dd4-9b0a-bfe97f442b85
07/10 14:25:29.216 20713,3 Cns_srv_addreplica: returns 0
07/10 14:25:29.217 20713,3 Cns_srv_endtrans: NS092 - endtrans request by /DC=org/DC=terena/DC=tcs/C=CZ/O=Czech Technical University in Prague/CN=Radek Ludacka 364520 (613,101) from ui1.egee.cesnet.cz
07/10 14:25:29.284 20713,3 Cns_srv_endtrans: returns 0
\end{lstlisting}

\begin{lstlisting}[label={log:lcgcr2},caption={lcg-cr --vo voce -d srm://srm.grid.sara.nl:8443/pnfs/grid.sara.nl/data/voce/ludacka/text\_file.txt -l lfn:/grid/voce/ludacka/text\_file.txt "file://\$PWD/text\_file.txt" - query to file that does not exists}]
07/09 10:24:29.797 20713,0 Cns_srv_lstat: NS092 - lstat request by /DC=org/DC=terena/DC=tcs/C=CZ/O=Czech Technical University in Prague/CN=Radek Ludacka 364520 (613,101) from ui1.egee.cesnet.cz
07/09 10:24:29.797 20713,0 Cns_srv_lstat: NS098 - lstat 0 /grid/voce/ludacka/text_file.txt
07/09 10:24:29.799 20713,0 Cns_srv_lstat: returns 2
\end{lstlisting}

\subsection{lfc-mkdir}

lfc-mkdir - creates a directory in LFC.

\begin{lstlisting}[label={log:lfcmkdir1},caption={lfc-mkdir /grid/voce/ludacka}]
07/09 10:22:13.423 20713,3 Cns_srv_mkdir: NS092 - mkdir request by /DC=org/DC=terena/DC=tcs/C=CZ/O=Czech Technical University in Prague/CN=Radek Ludacka 364520 (613,101) from ui1.egee.cesnet.cz
07/09 10:22:13.423 20713,3 Cns_srv_mkdir: NS098 - mkdir /grid/voce/ludacka  777 22
07/09 10:22:13.427 20713,3 Cns_srv_mkdir: returns 0
\end{lstlisting}

\begin{lstlisting}[label={log:lfcmkdir2},caption={lfc-mkdir /grid/voce/ludacka directory already exists}]
07/17 13:43:58.752 20713,8 Cns_srv_mkdir: NS092 - mkdir request by /DC=org/DC=terena/DC=tcs/C=CZ/O=Czech Technical University in Prague/CN=Radek Ludacka 364520 (613,101) from ui1.egee.cesnet.cz
07/17 13:43:58.752 20713,8 Cns_srv_mkdir: NS098 - mkdir /grid/voce/ludacka  777 22
07/17 13:43:58.755 20713,8 Cns_srv_mkdir: returns 17
\end{lstlisting}

\subsection{lfc-lr}

lfc-lr - lists number of replicas and their destinations.

\begin{lstlisting}[label={log:lr1},caption={lcg-lr lfn://grid/voce/ludacka/text\_file.txt}]
07/09 11:12:27.262 20713,3 Cns_srv_getreplica: NS092 - getreplica request by /DC=org/DC=terena/DC=tcs/C=CZ/O=Czech Technical University in Prague/CN=Radek Ludacka 364520 (613,101) from ui1.egee.cesnet.cz
07/09 11:12:27.262 20713,3 Cns_srv_getreplica: NS098 - getreplica /grid/voce/ludacka/text_file.txt 
07/09 11:12:27.264 20713,3 Cns_srv_getreplica: returns 0
\end{lstlisting}

\begin{lstlisting}[label={log:lr2},caption={lcg-lr lfn://grid/voce/ludacka/neexistfile.txt - query on file that does not exist}]
08/15 16:24:00.126 20713,1 Cns_srv_getreplica: NS092 - getreplica request by /DC=org/DC=terena/DC=tcs/C=CZ/O=Czech Technical University in Prague/CN=Radek Ludacka 364520 (613,101) from ui1.egee.cesnet.cz
08/15 16:24:00.126 20713,1 Cns_srv_getreplica: NS098 - getreplica /grid/voce/ludacka/neexistfile.txt
08/15 16:24:00.127 20713,1 Cns_srv_getreplica: returns 2
\end{lstlisting}

\newpage

\subsection{lcg-lg}

lcg-lg - gets the GUID for a given LFN or SURL.

\begin{lstlisting}[label={log:lg1},caption={lcg-lg --vo voce lfn:/grid/voce/ludacka}]
07/17 09:48:01.291 20713,1 Cns_srv_statg: NS092 - statg request by /DC=org/DC=terena/DC=tcs/C=CZ/O=Czech Technical University in Prague/CN=Radek Ludacka 364520 (613,101) from ui1.egee.cesnet.cz
07/17 09:48:01.291 20713,1 Cns_srv_statg: NS098 - statg /grid/voce/ludacka 
07/17 09:48:01.293 20713,1 Cns_srv_statg: returns 0
\end{lstlisting}

\begin{lstlisting}[label={log:lg2},caption={lcg-lg --vo voce lfn:/grid/voce/ludacka/text\_file.txt}]
07/17 09:49:34.952 20713,3 Cns_srv_statg: NS092 - statg request by /DC=org/DC=terena/DC=tcs/C=CZ/O=Czech Technical University in Prague/CN=Radek Ludacka 364520 (613,101) from ui1.egee.cesnet.cz
07/17 09:49:34.952 20713,3 Cns_srv_statg: NS098 - statg /grid/voce/ludacka/text_file.txt 
07/17 09:49:34.954 20713,3 Cns_srv_statg: returns 0
\end{lstlisting}

\subsection{lcg-la}

lcg-la - lists the aliases for a given LFN, GUID or SURL.

\begin{lstlisting}[label={log:la1},caption={lcg-la --vo voce guid:2045789b-6209-446f-83bc-ae32603b4ac7}]
07/17 09:46:04.940 20713,1 Cns_srv_getlinks: NS092 - getlinks request by /DC=org/DC=terena/DC=tcs/C=CZ/O=Czech Technical University in Prague/CN=Radek Ludacka 364520 (613,101) from ui1.egee.cesnet.cz
07/17 09:46:04.940 20713,1 Cns_srv_getlinks: NS098 - getlinks  2045789b-6209-446f-83bc-ae32603b4ac7
07/17 09:46:04.943 20713,1 Cns_srv_getlinks: returns 0
\end{lstlisting}

\newpage

\subsection{lcg-aa}

lcg-aa - adds an alias in RMC for a given GUID.

\begin{lstlisting}[label={log:aa1},caption={lcg-aa --vo voce guid:2045789b-6209-446f-83bc-ae32603b4ac7 /grid/voce/ludacka/text\_file10.txt}]
07/18 14:54:28.939 20713,0 Cns_srv_starttrans: NS092 - starttrans request by /DC=org/DC=terena/DC=tcs/C=CZ/O=Czech Technical University in Prague/CN=Radek Ludacka 364520 (613,101) from ui1.egee.cesnet.cz
07/18 14:54:28.939 20713,0 Cns_srv_starttrans: NS098 - starttrans (1.11.16-3)
07/18 14:54:28.939 20713,0 Cns_srv_starttrans: returns 0
07/18 14:54:28.939 20713,0 Cns_srv_statg: NS092 - statg request by /DC=org/DC=terena/DC=tcs/C=CZ/O=Czech Technical University in Prague/CN=Radek Ludacka 364520 (613,101) from ui1.egee.cesnet.cz
07/18 14:54:28.939 20713,0 Cns_srv_statg: NS098 - statg  2045789b-6209-446f-83bc-ae32603b4ac7
07/18 14:54:28.940 20713,0 Cns_srv_statg: returns 0
07/18 14:54:28.976 20713,0 Cns_srv_getpath: NS092 - getpath request by /DC=org/DC=terena/DC=tcs/C=CZ/O=Czech Technical University in Prague/CN=Radek Ludacka 364520 (613,101) from ui1.egee.cesnet.cz
07/18 14:54:28.978 20713,0 Cns_srv_getpath: returns 0
07/18 14:54:29.016 20713,0 Cns_srv_symlink: NS092 - symlink request by /DC=org/DC=terena/DC=tcs/C=CZ/O=Czech Technical University in Prague/CN=Radek Ludacka 364520 (613,101) from ui1.egee.cesnet.cz
07/18 14:54:29.016 20713,0 Cns_srv_symlink: NS098 - symlink /grid/voce/ludacka/text_file.txt /grid/voce/ludacka/text_file10.txt
07/18 14:54:29.048 20713,0 Cns_srv_symlink: returns 0
07/18 14:54:29.049 20713,0 Cns_srv_endtrans: NS092 - endtrans request by /DC=org/DC=terena/DC=tcs/C=CZ/O=Czech Technical University in Prague/CN=Radek Ludacka 364520 (613,101) from ui1.egee.cesnet.cz
07/18 14:54:29.134 20713,0 Cns_srv_endtrans: returns 0
\end{lstlisting}
\newpage
\subsection{lcg-cp}

lcg-cp - copies files from/to/between Storage Element(s), without registering anything in the file catalog.

\begin{lstlisting}[label={log:cp1},caption={lcg-cp --vo voce lfn:/grid/voce/ludacka/text\_file.txt file://\$PWD/text\_file1.txt}]
07/09 11:19:18.692 20713,1 Cns_srv_startsess: NS092 - startsess request by /DC=org/DC=terena/DC=tcs/C=CZ/O=Czech Technical University in Prague/CN=Radek Ludacka 364520 (613,101) from ui1.egee.cesnet.cz
07/09 11:19:18.692 20713,1 Cns_srv_startsess: NS098 - startsess (1.11.16-3)
07/09 11:19:18.692 20713,1 Cns_srv_startsess: returns 0
07/09 11:19:18.693 20713,1 Cns_srv_statg: NS092 - statg request by /DC=org/DC=terena/DC=tcs/C=CZ/O=Czech Technical University in Prague/CN=Radek Ludacka 364520 (613,101) from ui1.egee.cesnet.cz
07/09 11:19:18.693 20713,1 Cns_srv_statg: NS098 - statg /grid/voce/ludacka/text_file.txt 
07/09 11:19:18.695 20713,1 Cns_srv_statg: returns 0
07/09 11:19:18.735 20713,1 Cns_srv_getreplica: NS092 - getreplica request by /DC=org/DC=terena/DC=tcs/C=CZ/O=Czech Technical University in Prague/CN=Radek Ludacka 364520 (613,101) from ui1.egee.cesnet.cz
07/09 11:19:18.735 20713,1 Cns_srv_getreplica: NS098 - getreplica /grid/voce/ludacka/text_file.txt 4d56eecd-6b88-4262-8b64-957e1a62fb7d
07/09 11:19:18.738 20713,1 Cns_srv_getreplica: returns 0
07/09 11:19:18.774 20713,1 Cns_srv_endsess: NS092 - endsess request by /DC=org/DC=terena/DC=tcs/C=CZ/O=Czech Technical University in Prague/CN=Radek Ludacka 364520 (613,101) from ui1.egee.cesnet.cz
07/09 11:19:18.775 20713,1 Cns_srv_endsess: returns 0
\end{lstlisting}

\begin{lstlisting}[label={log:cp2},caption={lcg-cp --vo voce lfn:/grid/voce/ludacka/neexist\_file file://\$PWD/text\_file5.txt - lcg-cp to download not exists directory}]
07/17 15:03:01.852 20713,0 Cns_srv_startsess: NS092 - startsess request by /DC=org/DC=terena/DC=tcs/C=CZ/O=Czech Technical University in Prague/CN=Radek Ludacka 364520 (613,101) from ui1.egee.cesnet.cz
07/17 15:03:01.852 20713,0 Cns_srv_startsess: NS098 - startsess (1.11.16-3)
07/17 15:03:01.852 20713,0 Cns_srv_startsess: returns 0
07/17 15:03:01.852 20713,0 Cns_srv_statg: NS092 - statg request by /DC=org/DC=terena/DC=tcs/C=CZ/O=Czech Technical University in Prague/CN=Radek Ludacka 364520 (613,101) from ui1.egee.cesnet.cz
07/17 15:03:01.852 20713,0 Cns_srv_statg: NS098 - statg /grid/voce/ludacka/neexist_file 
07/17 15:03:01.853 20713,0 Cns_srv_statg: returns 2
07/17 15:03:01.853 20713,0 Cns_srv_endsess: NS092 - endsess request by /DC=org/DC=terena/DC=tcs/C=CZ/O=Czech Technical University in Prague/CN=Radek Ludacka 364520 (613,101) from ui1.egee.cesnet.cz
07/17 15:03:01.853 20713,0 Cns_srv_endsess: returns 0
\end{lstlisting}

\subsection{lcg-rep}

lcg-rep - copies a file from one Storage Element to another Storage.

\begin{lstlisting}[label={log:rep1},caption={lcg-rep --vo voce -d hephyse.oeaw.ac.at lfn:/grid/voce/ludacka/text\_file.txt}]
07/09 11:16:16.288 20713,2 Cns_srv_startsess: NS092 - startsess request by /DC=org/DC=terena/DC=tcs/C=CZ/O=Czech Technical University in Prague/CN=Radek Ludacka 364520 (613,101) from ui1.egee.cesnet.cz
07/09 11:16:16.288 20713,2 Cns_srv_startsess: NS098 - startsess (1.11.16-3)
07/09 11:16:16.288 20713,2 Cns_srv_startsess: returns 0
07/09 11:16:16.288 20713,2 Cns_srv_statg: NS092 - statg request by /DC=org/DC=terena/DC=tcs/C=CZ/O=Czech Technical University in Prague/CN=Radek Ludacka 364520 (613,101) from ui1.egee.cesnet.cz
07/09 11:16:16.288 20713,2 Cns_srv_statg: NS098 - statg /grid/voce/ludacka/text_file.txt 
07/09 11:16:16.291 20713,2 Cns_srv_statg: returns 0
07/09 11:16:16.329 20713,2 Cns_srv_getreplica: NS092 - getreplica request by /DC=org/DC=terena/DC=tcs/C=CZ/O=Czech Technical University in Prague/CN=Radek Ludacka 364520 (613,101) from ui1.egee.cesnet.cz
07/09 11:16:16.329 20713,2 Cns_srv_getreplica: NS098 - getreplica /grid/voce/ludacka/text_file.txt 4d56eecd-6b88-4262-8b64-957e1a62fb7d
07/09 11:16:16.332 20713,2 Cns_srv_getreplica: returns 0
07/09 11:16:16.369 20713,2 Cns_srv_endsess: NS092 - endsess request by /DC=org/DC=terena/DC=tcs/C=CZ/O=Czech Technical University in Prague/CN=Radek Ludacka 364520 (613,101) from ui1.egee.cesnet.cz
07/09 11:16:16.369 20713,2 Cns_srv_endsess: returns 0
07/09 11:16:30.400 20713,4 Cns_srv_starttrans: NS092 - starttrans request by /DC=org/DC=terena/DC=tcs/C=CZ/O=Czech Technical University in Prague/CN=Radek Ludacka 364520 (613,101) from ui1.egee.cesnet.cz
07/09 11:16:30.400 20713,4 Cns_srv_starttrans: NS098 - starttrans (1.11.16-3)
07/09 11:16:30.400 20713,4 Cns_srv_starttrans: returns 0
07/09 11:16:30.400 20713,4 Cns_srv_statg: NS092 - statg request by /DC=org/DC=terena/DC=tcs/C=CZ/O=Czech Technical University in Prague/CN=Radek Ludacka 364520 (613,101) from ui1.egee.cesnet.cz
07/09 11:16:30.400 20713,4 Cns_srv_statg: NS098 - statg /grid/voce/ludacka/text_file.txt 4d56eecd-6b88-4262-8b64-957e1a62fb7d
07/09 11:16:30.403 20713,4 Cns_srv_statg: returns 0
07/09 11:16:30.441 20713,4 Cns_srv_addreplica: NS092 - addreplica request by /DC=org/DC=terena/DC=tcs/C=CZ/O=Czech Technical University in Prague/CN=Radek Ludacka 364520 (613,101) from ui1.egee.cesnet.cz
07/09 11:16:30.441 20713,4 Cns_srv_addreplica: NS098 - addreplica 4d56eecd-6b88-4262-8b64-957e1a62fb7d hephyse.oeaw.ac.at srm://hephyse.oeaw.ac.at/dpm/oeaw.ac.at/home/voce/generated/2013-07-09/filef16ea56a-efd1-40f1-a045-74fe6cf293f8
07/09 11:16:30.460 20713,4 Cns_srv_addreplica: returns 0
07/09 11:16:30.461 20713,4 Cns_srv_endtrans: NS092 - endtrans request by /DC=org/DC=terena/DC=tcs/C=CZ/O=Czech Technical University in Prague/CN=Radek Ludacka 364520 (613,101) from ui1.egee.cesnet.cz
\end{lstlisting}

\begin{lstlisting}[label={log:rep2},caption={lcg-rep --vo voce -d darkmass.wcss.wroc.pl lfn:/grid/voce/ludacka/text\_file.txt - replicate file that does not exist}]
07/10 18:49:37.299 20713,2 Cns_srv_startsess: NS092 - startsess request by /DC=org/DC=terena/DC=tcs/C=CZ/O=Czech Technical University in Prague/CN=Radek Ludacka 364520 (613,101) from ui1.egee.cesnet.cz
07/10 18:49:37.299 20713,2 Cns_srv_startsess: NS098 - startsess (1.11.16-3)
07/10 18:49:37.299 20713,2 Cns_srv_startsess: returns 0
07/10 18:49:37.300 20713,2 Cns_srv_statg: NS092 - statg request by /DC=org/DC=terena/DC=tcs/C=CZ/O=Czech Technical University in Prague/CN=Radek Ludacka 364520 (613,101) from ui1.egee.cesnet.cz
07/10 18:49:37.300 20713,2 Cns_srv_statg: NS098 - statg /grid/voce/ludacka/text_file.txt 
07/10 18:49:37.302 20713,2 Cns_srv_statg: returns 2
07/10 18:49:37.302 20713,2 Cns_srv_endsess: NS092 - endsess request by /DC=org/DC=terena/DC=tcs/C=CZ/O=Czech Technical University in Prague/CN=Radek Ludacka 364520 (613,101) from ui1.egee.cesnet.cz
07/10 18:49:37.302 20713,2 Cns_srv_endsess: returns 0
\end{lstlisting}

\begin{lstlisting}[label={log:rep3},caption={lcg-rep --vo voce -d some.notexist.site lfn:/grid/voce/ludacka/text\_file.txt - replicate to site that does not exist}]
07/17 08:30:48.815 20713,5 Cns_srv_startsess: NS092 - startsess request by /DC=org/DC=terena/DC=tcs/C=CZ/O=Czech Technical University in Prague/CN=Radek Ludacka 364520 (613,101) from ui1.egee.cesnet.cz
07/17 08:30:48.815 20713,5 Cns_srv_startsess: NS098 - startsess (1.11.16-3)
07/17 08:30:48.815 20713,5 Cns_srv_startsess: returns 0
07/17 08:30:48.816 20713,5 Cns_srv_statg: NS092 - statg request by /DC=org/DC=terena/DC=tcs/C=CZ/O=Czech Technical University in Prague/CN=Radek Ludacka 364520 (613,101) from ui1.egee.cesnet.cz
07/17 08:30:48.816 20713,5 Cns_srv_statg: NS098 - statg /grid/voce/ludacka/text_file.txt 
07/17 08:30:48.817 20713,5 Cns_srv_statg: returns 0
07/17 08:30:48.859 20713,5 Cns_srv_getreplica: NS092 - getreplica request by /DC=org/DC=terena/DC=tcs/C=CZ/O=Czech Technical University in Prague/CN=Radek Ludacka 364520 (613,101) from ui1.egee.cesnet.cz
07/17 08:30:48.859 20713,5 Cns_srv_getreplica: NS098 - getreplica /grid/voce/ludacka/text_file.txt 2045789b-6209-446f-83bc-ae32603b4ac7
07/17 08:30:48.866 20713,5 Cns_srv_getreplica: returns 0
07/17 08:30:48.904 20713,5 Cns_srv_endsess: NS092 - endsess request by /DC=org/DC=terena/DC=tcs/C=CZ/O=Czech Technical University in Prague/CN=Radek Ludacka 364520 (613,101) from ui1.egee.cesnet.cz
07/17 08:30:48.904 20713,5 Cns_srv_endsess: returns 0
\end{lstlisting}

\subsection{lcg-del}

lcg-del - deletes files (SURLs or LFNs)

\begin{lstlisting}[label={log:del1},caption={lcg-del --vo voce srm://dpm1.egee.cesnet.cz/dpm/cesnet.cz/home/voce/generated/2013-07-16/filecdd9d191-054b-4779-8d0f-c0b91fc564fb - delete only one replica}]
07/18 09:52:13.680 20713,15 Cns_srv_startsess: NS092 - startsess request by /DC=org/DC=terena/DC=tcs/C=CZ/O=Czech Technical University in Prague/CN=Radek Ludacka 364520 (613,101) from ui1.egee.cesnet.cz
07/18 09:52:13.680 20713,15 Cns_srv_startsess: NS098 - startsess (1.11.16-3)
07/18 09:52:13.680 20713,15 Cns_srv_startsess: returns 0
07/18 09:52:13.680 20713,15 Cns_srv_statr: NS092 - statr request by /DC=org/DC=terena/DC=tcs/C=CZ/O=Czech Technical University in Prague/CN=Radek Ludacka 364520 (613,101) from ui1.egee.cesnet.cz
07/18 09:52:13.680 20713,15 Cns_srv_statr: NS098 - statr srm://dpm1.egee.cesnet.cz/dpm/cesnet.cz/home/voce/generated/2013-07-16/filecdd9d191-054b-4779-8d0f-c0b91fc564fb
07/18 09:52:13.682 20713,15 Cns_srv_statr: returns 0
07/18 09:52:13.720 20713,15 Cns_srv_getlinks: NS092 - getlinks request by /DC=org/DC=terena/DC=tcs/C=CZ/O=Czech Technical University in Prague/CN=Radek Ludacka 364520 (613,101) from ui1.egee.cesnet.cz
07/18 09:52:13.720 20713,15 Cns_srv_getlinks: NS098 - getlinks  2045789b-6209-446f-83bc-ae32603b4ac7
07/18 09:52:13.743 20713,15 Cns_srv_getlinks: returns 0
07/18 09:52:13.784 20713,15 Cns_srv_access: NS092 - access request by /DC=org/DC=terena/DC=tcs/C=CZ/O=Czech Technical University in Prague/CN=Radek Ludacka 364520 (613,101) from ui1.egee.cesnet.cz
07/18 09:52:13.784 20713,15 Cns_srv_access: NS098 - access 6 /grid/voce/ludacka/text_file.txt
07/18 09:52:13.786 20713,15 Cns_srv_access: returns 0
07/18 09:52:13.787 20713,15 Cns_srv_endsess: NS092 - endsess request by /DC=org/DC=terena/DC=tcs/C=CZ/O=Czech Technical University in Prague/CN=Radek Ludacka 364520 (613,101) from ui1.egee.cesnet.cz
07/18 09:52:13.787 20713,15 Cns_srv_endsess: returns 0
07/18 09:52:16.312 20713,27 Cns_srv_startsess: NS092 - startsess request by /DC=org/DC=terena/DC=tcs/C=CZ/O=Czech Technical University in Prague/CN=Radek Ludacka 364520 (613,101) from ui1.egee.cesnet.cz
07/18 09:52:16.312 20713,27 Cns_srv_startsess: NS098 - startsess ()
07/18 09:52:16.312 20713,27 Cns_srv_startsess: returns 0
07/18 09:52:16.313 20713,27 Cns_srv_delreplica: NS092 - delreplica request by /DC=org/DC=terena/DC=tcs/C=CZ/O=Czech Technical University in Prague/CN=Radek Ludacka 364520 (613,101) from ui1.egee.cesnet.cz
07/18 09:52:16.313 20713,27 Cns_srv_delreplica: NS098 - delreplica 2045789b-6209-446f-83bc-ae32603b4ac7 srm://dpm1.egee.cesnet.cz/dpm/cesnet.cz/home/voce/generated/2013-07-16/filecdd9d191-054b-4779-8d0f-c0b91fc564fb
07/18 09:52:16.316 20713,27 Cns_srv_delreplica: returns 0
07/18 09:52:16.359 20713,27 Cns_srv_getreplica: NS092 - getreplica request by /DC=org/DC=terena/DC=tcs/C=CZ/O=Czech Technical University in Prague/CN=Radek Ludacka 364520 (613,101) from ui1.egee.cesnet.cz
07/18 09:52:16.359 20713,27 Cns_srv_getreplica: NS098 - getreplica  2045789b-6209-446f-83bc-ae32603b4ac7
07/18 09:52:16.360 20713,27 Cns_srv_getreplica: returns 0
07/18 09:52:16.400 20713,27 Cns_srv_endsess: NS092 - endsess request by /DC=org/DC=terena/DC=tcs/C=CZ/O=Czech Technical University in Prague/CN=Radek Ludacka 364520 (613,101) from ui1.egee.cesnet.cz
07/18 09:52:16.400 20713,27 Cns_srv_endsess: returns 0
\end{lstlisting}

\begin{lstlisting}[label={log:del3},caption={lcg-del -a lfn:/grid/voce/ludacka/text\_file.txt - deletes all replicas and alisses. In this example: deletes 2 replicas}] 
07/19 10:43:07.460 20713,0 Cns_srv_startsess: NS092 - startsess request by /DC=org/DC=terena/DC=tcs/C=CZ/O=Czech Technical University in Prague/CN=Radek Ludacka 364520 (613,101) from ui1.egee.cesnet.cz
07/19 10:43:07.460 20713,0 Cns_srv_startsess: NS098 - startsess (1.11.16-3)
07/19 10:43:07.460 20713,0 Cns_srv_startsess: returns 0
07/19 10:43:07.460 20713,0 Cns_srv_statg: NS092 - statg request by /DC=org/DC=terena/DC=tcs/C=CZ/O=Czech Technical University in Prague/CN=Radek Ludacka 364520 (613,101) from ui1.egee.cesnet.cz
07/19 10:43:07.460 20713,0 Cns_srv_statg: NS098 - statg /grid/voce/ludacka/text_file.txt 
07/19 10:43:07.463 20713,0 Cns_srv_statg: returns 0
07/19 10:43:07.501 20713,0 Cns_srv_getreplica: NS092 - getreplica request by /DC=org/DC=terena/DC=tcs/C=CZ/O=Czech Technical University in Prague/CN=Radek Ludacka 364520 (613,101) from ui1.egee.cesnet.cz
07/19 10:43:07.501 20713,0 Cns_srv_getreplica: NS098 - getreplica /grid/voce/ludacka/text_file.txt 78eda959-9af0-4423-9b01-3eade1cffe31
07/19 10:43:07.504 20713,0 Cns_srv_getreplica: returns 0
07/19 10:43:07.541 20713,0 Cns_srv_endsess: NS092 - endsess request by /DC=org/DC=terena/DC=tcs/C=CZ/O=Czech Technical University in Prague/CN=Radek Ludacka 364520 (613,101) from ui1.egee.cesnet.cz
07/19 10:43:07.541 20713,0 Cns_srv_endsess: returns 0
07/19 10:43:08.609 20713,1 Cns_srv_startsess: NS092 - startsess request by /DC=org/DC=terena/DC=tcs/C=CZ/O=Czech Technical University in Prague/CN=Radek Ludacka 364520 (613,101) from ui1.egee.cesnet.cz
07/19 10:43:08.609 20713,1 Cns_srv_startsess: NS098 - startsess (1.11.16-3)
07/19 10:43:08.609 20713,1 Cns_srv_startsess: returns 0
07/19 10:43:08.609 20713,1 Cns_srv_statg: NS092 - statg request by /DC=org/DC=terena/DC=tcs/C=CZ/O=Czech Technical University in Prague/CN=Radek Ludacka 364520 (613,101) from ui1.egee.cesnet.cz
07/19 10:43:08.609 20713,1 Cns_srv_statg: NS098 - statg /grid/voce/ludacka/text_file.txt 
07/19 10:43:08.612 20713,1 Cns_srv_statg: returns 0
07/19 10:43:08.645 20713,1 Cns_srv_getreplica: NS092 - getreplica request by /DC=org/DC=terena/DC=tcs/C=CZ/O=Czech Technical University in Prague/CN=Radek Ludacka 364520 (613,101) from ui1.egee.cesnet.cz
07/19 10:43:08.645 20713,1 Cns_srv_getreplica: NS098 - getreplica /grid/voce/ludacka/text_file.txt 78eda959-9af0-4423-9b01-3eade1cffe31
07/19 10:43:08.647 20713,1 Cns_srv_getreplica: returns 0
07/19 10:43:08.685 20713,1 Cns_srv_endsess: NS092 - endsess request by /DC=org/DC=terena/DC=tcs/C=CZ/O=Czech Technical University in Prague/CN=Radek Ludacka 364520 (613,101) from ui1.egee.cesnet.cz
07/19 10:43:08.685 20713,1 Cns_srv_endsess: returns 0
07/19 10:43:09.750 20713,3 Cns_srv_access: NS092 - access request by /DC=org/DC=terena/DC=tcs/C=CZ/O=Czech Technical University in Prague/CN=Radek Ludacka 364520 (613,101) from ui1.egee.cesnet.cz
07/19 10:43:09.750 20713,3 Cns_srv_access: NS098 - access 6 /grid/voce/ludacka/text_file.txt
07/19 10:43:09.751 20713,3 Cns_srv_access: returns 0
07/19 10:43:17.260 20713,3 Cns_srv_startsess: NS092 - startsess request by /DC=org/DC=terena/DC=tcs/C=CZ/O=Czech Technical University in Prague/CN=Radek Ludacka 364520 (613,101) from ui1.egee.cesnet.cz
07/19 10:43:17.260 20713,3 Cns_srv_startsess: NS098 - startsess ()
07/19 10:43:17.260 20713,3 Cns_srv_startsess: returns 0
07/19 10:43:17.261 20713,3 Cns_srv_delreplica: NS092 - delreplica request by /DC=org/DC=terena/DC=tcs/C=CZ/O=Czech Technical University in Prague/CN=Radek Ludacka 364520 (613,101) from ui1.egee.cesnet.cz
07/19 10:43:17.261 20713,3 Cns_srv_delreplica: NS098 - delreplica 78eda959-9af0-4423-9b01-3eade1cffe31 srm://dpm1.egee.cesnet.cz/dpm/cesnet.cz/home/voce/generated/2013-07-19/file290083f4-7c35-4f68-9ef0-9d6cab6eea3e
07/19 10:43:17.265 20713,3 Cns_srv_delreplica: returns 0
07/19 10:43:17.351 20713,3 Cns_srv_getreplica: NS092 - getreplica request by /DC=org/DC=terena/DC=tcs/C=CZ/O=Czech Technical University in Prague/CN=Radek Ludacka 364520 (613,101) from ui1.egee.cesnet.cz
07/19 10:43:17.351 20713,3 Cns_srv_getreplica: NS098 - getreplica  78eda959-9af0-4423-9b01-3eade1cffe31
07/19 10:43:17.352 20713,3 Cns_srv_getreplica: returns 0
07/19 10:43:17.389 20713,3 Cns_srv_endsess: NS092 - endsess request by /DC=org/DC=terena/DC=tcs/C=CZ/O=Czech Technical University in Prague/CN=Radek Ludacka 364520 (613,101) from ui1.egee.cesnet.cz
07/19 10:43:17.389 20713,3 Cns_srv_endsess: returns 0
07/19 10:43:23.025 20713,1 Cns_srv_startsess: returns 0
07/19 10:43:23.025 20713,1 Cns_srv_delreplica: NS092 - delreplica request by /DC=org/DC=terena/DC=tcs/C=CZ/O=Czech Technical University in Prague/CN=Radek Ludacka 364520 (613,101) from ui1.egee.cesnet.cz
07/19 10:43:23.025 20713,1 Cns_srv_delreplica: NS098 - delreplica 78eda959-9af0-4423-9b01-3eade1cffe31 srm://darkmass.wcss.wroc.pl/dpm/wcss.wroc.pl/home/voce/generated/2013-07-19/file71c76adf-b0a3-4ade-b292-55ee050f4e54
07/19 10:43:23.029 20713,1 Cns_srv_delreplica: returns 0
07/19 10:43:23.102 20713,1 Cns_srv_getreplica: NS092 - getreplica request by /DC=org/DC=terena/DC=tcs/C=CZ/O=Czech Technical University in Prague/CN=Radek Ludacka 364520 (613,101) from ui1.egee.cesnet.cz
07/19 10:43:23.102 20713,1 Cns_srv_getreplica: NS098 - getreplica  78eda959-9af0-4423-9b01-3eade1cffe31
07/19 10:43:23.103 20713,1 Cns_srv_getreplica: returns 0
07/19 10:43:23.141 20713,1 Cns_srv_getlinks: NS092 - getlinks request by /DC=org/DC=terena/DC=tcs/C=CZ/O=Czech Technical University in Prague/CN=Radek Ludacka 364520 (613,101) from ui1.egee.cesnet.cz
07/19 10:43:23.141 20713,1 Cns_srv_getlinks: NS098 - getlinks  78eda959-9af0-4423-9b01-3eade1cffe31
07/19 10:43:23.144 20713,1 Cns_srv_getlinks: returns 0
07/19 10:43:23.181 20713,1 Cns_srv_unlink: NS092 - unlink request by /DC=org/DC=terena/DC=tcs/C=CZ/O=Czech Technical University in Prague/CN=Radek Ludacka 364520 (613,101) from ui1.egee.cesnet.cz
07/19 10:43:23.181 20713,1 Cns_srv_unlink: NS098 - unlink /grid/voce/ludacka/text_file.txt
07/19 10:43:23.190 20713,1 Cns_srv_unlink: returns 0
07/19 10:43:23.261 20713,1 Cns_srv_endsess: NS092 - endsess request by /DC=org/DC=terena/DC=tcs/C=CZ/O=Czech Technical University in Prague/CN=Radek Ludacka 364520 (613,101) from ui1.egee.cesnet.cz
07/19 10:43:23.261 20713,1 Cns_srv_endsess: returns 0
\end{lstlisting}

\begin{lstlisting}[label={log:del4},caption={lcg-del --vo voce srm://dpm1.egee.cesnet.cz/dpm/cesnet.cz/home/voce/generated/2013-07-18/fileda937a3c-4112-4d56-a34e-9866967cfe44 - deletes the last replica and all its links (2)}]
07/18 15:46:24.732 20713,1 Cns_srv_startsess: NS092 - startsess request by /DC=org/DC=terena/DC=tcs/C=CZ/O=Czech Technical University in Prague/CN=Radek Ludacka 364520 (613,101) from ui1.egee.cesnet.cz
07/18 15:46:24.732 20713,1 Cns_srv_startsess: NS098 - startsess (1.11.16-3)
07/18 15:46:24.732 20713,1 Cns_srv_startsess: returns 0
07/18 15:46:24.733 20713,1 Cns_srv_statr: NS092 - statr request by /DC=org/DC=terena/DC=tcs/C=CZ/O=Czech Technical University in Prague/CN=Radek Ludacka 364520 (613,101) from ui1.egee.cesnet.cz
07/18 15:46:24.733 20713,1 Cns_srv_statr: NS098 - statr srm://dpm1.egee.cesnet.cz/dpm/cesnet.cz/home/voce/generated/2013-07-18/fileda937a3c-4112-4d56-a34e-9866967cfe44
07/18 15:46:24.734 20713,1 Cns_srv_statr: returns 0
07/18 15:46:24.773 20713,1 Cns_srv_getlinks: NS092 - getlinks request by /DC=org/DC=terena/DC=tcs/C=CZ/O=Czech Technical University in Prague/CN=Radek Ludacka 364520 (613,101) from ui1.egee.cesnet.cz
07/18 15:46:24.773 20713,1 Cns_srv_getlinks: NS098 - getlinks  08043bf1-37e7-4f53-b642-cb474db58f90
07/18 15:46:24.779 20713,1 Cns_srv_getlinks: returns 0
07/18 15:46:24.815 20713,1 Cns_srv_access: NS092 - access request by /DC=org/DC=terena/DC=tcs/C=CZ/O=Czech Technical University in Prague/CN=Radek Ludacka 364520 (613,101) from ui1.egee.cesnet.cz
07/18 15:46:24.815 20713,1 Cns_srv_access: NS098 - access 6 /grid/voce/ludacka/text_file.txt
07/18 15:46:24.817 20713,1 Cns_srv_access: returns 0
07/18 15:46:24.818 20713,1 Cns_srv_endsess: NS092 - endsess request by /DC=org/DC=terena/DC=tcs/C=CZ/O=Czech Technical University in Prague/CN=Radek Ludacka 364520 (613,101) from ui1.egee.cesnet.cz
07/18 15:46:24.818 20713,1 Cns_srv_endsess: returns 0
07/18 15:46:27.326 20713,2 Cns_srv_startsess: NS092 - startsess request by /DC=org/DC=terena/DC=tcs/C=CZ/O=Czech Technical University in Prague/CN=Radek Ludacka 364520 (613,101) from ui1.egee.cesnet.cz
07/18 15:46:27.326 20713,2 Cns_srv_startsess: NS098 - startsess ()
07/18 15:46:27.326 20713,2 Cns_srv_startsess: returns 0
07/18 15:46:27.326 20713,2 Cns_srv_delreplica: NS092 - delreplica request by /DC=org/DC=terena/DC=tcs/C=CZ/O=Czech Technical University in Prague/CN=Radek Ludacka 364520 (613,101) from ui1.egee.cesnet.cz
07/18 15:46:27.326 20713,2 Cns_srv_delreplica: NS098 - delreplica 08043bf1-37e7-4f53-b642-cb474db58f90 srm://dpm1.egee.cesnet.cz/dpm/cesnet.cz/home/voce/generated/2013-07-18/fileda937a3c-4112-4d56-a34e-9866967cfe44
07/18 15:46:27.329 20713,2 Cns_srv_delreplica: returns 0
07/18 15:46:27.366 20713,2 Cns_srv_getreplica: NS092 - getreplica request by /DC=org/DC=terena/DC=tcs/C=CZ/O=Czech Technical University in Prague/CN=Radek Ludacka 364520 (613,101) from ui1.egee.cesnet.cz
07/18 15:46:27.366 20713,2 Cns_srv_getreplica: NS098 - getreplica  08043bf1-37e7-4f53-b642-cb474db58f90
07/18 15:46:27.367 20713,2 Cns_srv_getreplica: returns 0
07/18 15:46:27.405 20713,2 Cns_srv_getlinks: NS092 - getlinks request by /DC=org/DC=terena/DC=tcs/C=CZ/O=Czech Technical University in Prague/CN=Radek Ludacka 364520 (613,101) from ui1.egee.cesnet.cz
07/18 15:46:27.406 20713,2 Cns_srv_getlinks: NS098 - getlinks  08043bf1-37e7-4f53-b642-cb474db58f90
07/18 15:46:27.409 20713,2 Cns_srv_getlinks: returns 0
07/18 15:46:27.446 20713,2 Cns_srv_unlink: NS092 - unlink request by /DC=org/DC=terena/DC=tcs/C=CZ/O=Czech Technical University in Prague/CN=Radek Ludacka 364520 (613,101) from ui1.egee.cesnet.cz
07/18 15:46:27.446 20713,2 Cns_srv_unlink: NS098 - unlink /grid/voce/ludacka/text_file.txt
07/18 15:46:27.455 20713,2 Cns_srv_unlink: returns 0
07/18 15:46:27.528 20713,2 Cns_srv_unlink: NS092 - unlink request by /DC=org/DC=terena/DC=tcs/C=CZ/O=Czech Technical University in Prague/CN=Radek Ludacka 364520 (613,101) from ui1.egee.cesnet.cz
07/18 15:46:27.528 20713,2 Cns_srv_unlink: NS098 - unlink /grid/voce/ludacka/text_file8.txt
07/18 15:46:27.537 20713,2 Cns_srv_unlink: returns 0
07/18 15:46:27.577 20713,2 Cns_srv_endsess: NS092 - endsess request by /DC=org/DC=terena/DC=tcs/C=CZ/O=Czech Technical University in Prague/CN=Radek Ludacka 364520 (613,101) from ui1.egee.cesnet.cz
07/18 15:46:27.577 20713,2 Cns_srv_endsess: returns 0
\end{lstlisting}

\begin{lstlisting}[label={log:del5},caption={lcg-del -a lfn:/grid/voce/ludacrad/text\_file.txt - try to delete file that does not exist}]
07/09 11:22:05.671 20713,4 Cns_srv_startsess: NS092 - startsess request by /DC=org/DC=terena/DC=tcs/C=CZ/O=Czech Technical University in Prague/CN=Radek Ludacka 364520 (613,101) from ui1.egee.cesnet.cz
07/09 11:22:05.671 20713,4 Cns_srv_startsess: NS098 - startsess (1.11.16-3)
07/09 11:22:05.671 20713,4 Cns_srv_startsess: returns 0
07/09 11:22:05.671 20713,4 Cns_srv_statg: NS092 - statg request by /DC=org/DC=terena/DC=tcs/C=CZ/O=Czech Technical University in Prague/CN=Radek Ludacka 364520 (613,101) from ui1.egee.cesnet.cz
07/09 11:22:05.671 20713,4 Cns_srv_statg: NS098 - statg /grid/voce/ludacrad/text_file.txt
07/09 11:22:05.673 20713,4 Cns_srv_statg: returns 2
07/09 11:22:05.673 20713,4 Cns_srv_endsess: NS092 - endsess request by /DC=org/DC=terena/DC=tcs/C=CZ/O=Czech Technical University in Prague/CN=Radek Ludacka 364520 (613,101) from ui1.egee.cesnet.cz
07/09 11:22:05.673 20713,4 Cns_srv_endsess: returns 0
\end{lstlisting}

Writing several lines is necessary to describe lcg-del command because of its complexity. The number of result lines written down to the log file is dependent on how many replicas and how many links have been created before lcg-del invocation. lcg-del command has two basic variants: the first that deletes only one replica (with --vo option and SURL path) and the second one (with -a option and LFN path) that deletes all replicas from all storage elements and all links. The last lines define all variants that could happen therefore only the last few lines (sequences) will be described in this review.

The first variant deletes only one replica if more exists (sequence: startsess, getreplica, delreplica, endsess). If command deletes the last  replica then it also removes all its links (sequence: startsess, getreplica, delreplica, getlinks, unlink, endsess) where number of invoked unlink command is depends on the number of existing links.

The second variant deletes all replicas and links - sequence: startsess, getreplica, delreplica, endsess is written down to the log file for each deleted replica after the same sequence (startsess, getreplica, delreplica, getlinks, unlink, endsess) as in the first variant is written down to the log file.

\subsection{lfc-rm}

lfc-rm removes LFC files or directories from the name server.

\begin{lstlisting}[label={log:rm},caption={lfc-rm -r /grid/voce/ludacka2}]
07/10 19:28:38.919 20713,5 Cns_srv_lstat: NS092 - lstat request by /DC=org/DC=terena/DC=tcs/C=CZ/O=Czech Technical University in Prague/CN=Radek Ludacka 364520 (613,101) from ui1.egee.cesnet.cz
07/10 19:28:38.919 20713,5 Cns_srv_lstat: NS098 - lstat 0 /grid/voce/ludacka2
07/10 19:28:38.920 20713,5 Cns_srv_lstat: returns 0
07/10 19:28:38.989 20713,1 Cns_srv_access: NS092 - access request by /DC=org/DC=terena/DC=tcs/C=CZ/O=Czech Technical University in Prague/CN=Radek Ludacka 364520 (613,101) from ui1.egee.cesnet.cz
07/10 19:28:38.989 20713,1 Cns_srv_access: NS098 - access 2 /grid/voce/ludacka2
07/10 19:28:38.990 20713,1 Cns_srv_access: returns 0
07/10 19:28:39.054 20713,1 Cns_srv_opendir: NS092 - opendir request by /DC=org/DC=terena/DC=tcs/C=CZ/O=Czech Technical University in Prague/CN=Radek Ludacka 364520 (613,101) from ui1.egee.cesnet.cz
07/10 19:28:39.054 20713,1 Cns_srv_opendir: NS098 - opendir /grid/voce/ludacka2 
07/10 19:28:39.056 20713,1 Cns_srv_opendir: returns 0
07/10 19:28:40.160 20713,4 Cns_srv_chdir: NS092 - chdir request by /DC=org/DC=terena/DC=tcs/C=CZ/O=Czech Technical University in Prague/CN=Radek Ludacka 364520 (613,101) from ui1.egee.cesnet.cz
07/10 19:28:40.160 20713,4 Cns_srv_chdir: NS098 - chdir /grid/voce/ludacka2
07/10 19:28:40.163 20713,4 Cns_srv_chdir: returns 0
07/10 19:28:40.163 20713,1 Cns_srv_readdir: NS092 - readdir request by /DC=org/DC=terena/DC=tcs/C=CZ/O=Czech Technical University in Prague/CN=Radek Ludacka 364520 (613,101) from ui1.egee.cesnet.cz
07/10 19:28:40.165 20713,1 Cns_srv_readdir: returns 0
07/10 19:28:40.232 20713,1 Cns_srv_readdir: NS092 - closedir request by /DC=org/DC=terena/DC=tcs/C=CZ/O=Czech Technical University in Prague/CN=Radek Ludacka 364520 (613,101) from ui1.egee.cesnet.cz
07/10 19:28:40.233 20713,1 Cns_srv_readdir: returns 0
07/10 19:28:41.351 20713,1 Cns_srv_chdir: NS092 - chdir request by /DC=org/DC=terena/DC=tcs/C=CZ/O=Czech Technical University in Prague/CN=Radek Ludacka 364520 (613,101) from ui1.egee.cesnet.cz
07/10 19:28:41.351 20713,1 Cns_srv_chdir: NS098 - chdir ..
07/10 19:28:41.352 20713,1 Cns_srv_chdir: returns 0
07/10 19:28:41.420 20713,1 Cns_srv_rmdir: NS092 - rmdir request by /DC=org/DC=terena/DC=tcs/C=CZ/O=Czech Technical University in Prague/CN=Radek Ludacka 364520 (613,101) from ui1.egee.cesnet.cz
07/10 19:28:41.420 20713,1 Cns_srv_rmdir: NS098 - rmdir /grid/voce/ludacka2
07/10 19:28:41.426 20713,1 Cns_srv_rmdir: returns 0
\end{lstlisting}

\begin{lstlisting}[label={log:rm2},caption={lfc-rm /grid/voce/ludacka2 - error: Is a directory}]
07/10 19:24:12.473 20713,0 Cns_srv_lstat: NS092 - lstat request by /DC=org/DC=terena/DC=tcs/C=CZ/O=Czech Technical University in Prague/CN=Radek Ludacka 364520 (613,101) from ui1.egee.cesnet.cz
07/10 19:24:12.473 20713,0 Cns_srv_lstat: NS098 - lstat 0 /grid/voce/ludacka2
07/10 19:24:12.475 20713,0 Cns_srv_lstat: returns 0
\end{lstlisting}

\newpage

\subsection{Log lines written down by service or when an error occurs}

\begin{lstlisting}[label={log:utime},caption={Cns\_srv\_utime - invoked by nagios service}]
11/30 07:45:39.237 16392,0 Cns_srv_utime: NS092 - utime request by /DC=cz/DC=cesnet-ca/O=Institute of Physics of the Academy of Sciences of the CR/CN=Jan Svec (557,120,138,143) from nagios.egee.cesnet.cz
11/30 07:45:39.237 16392,0 Cns_srv_utime: NS098 - utime /grid/ops/file-lfc-probe-lfc1.egee.cesnet.cz 0
11/30 07:45:39.240 16392,0 Cns_srv_utime: returns 0
\end{lstlisting}

\begin{lstlisting}[label={log:timeout},caption={Cns\_serv - timed out}]
11/30 05:07:09.707 16392,1 Cns_serv: NS002 - netread error : Timed out
11/30 05:07:09.707 16392,1 Cns_serv: [147.231.25.105] (nat105.farm.particle.cz): Failure getting the request: Timed out
\end{lstlisting}

\begin{lstlisting}[label={log:proxyproblem},caption={Cns\_serv - proxy problem}]
11/30 04:21:25.123 16392,1 Cns_serv: [132.195.125.93] (wn083.pleiades.uni-wuppertal.de): Could not establish an authenticated connection: server_establish_context_ext: The client itself detected a problem with the user proxy, it was probably missing or expired !
\end{lstlisting}

\begin{lstlisting}[label={log:pingdb},caption={Cns\_pingdb - MySql query error}]
11/30 07:58:36.747 16392,20 Cns_pingdb: mysql_query error: MySQL server has gone away
11/30 07:58:36.747 16392,20 Cns_pingdb: Trying to reconnect\
\end{lstlisting}

\newpage

\begin{lstlisting}[label={log:brokenpipe},caption={sendrep - Broken pipe}]
lines of lfc-ls log ...
07/18 12:55:30.006 20713,3 Cns_srv_readdir: returns 0
07/18 12:57:32.210 20713,3 Cns_serv: NS002 - netread error : Connection reset by peer
07/18 12:57:32.210 20713,3 Cns_srv_readdir: NS092 - closedir request by /DC=es/DC=irisgrid/O=ugr/CN=gines.rubio (491,117,104) from wario.univ-lille1.fr
07/18 12:57:32.212 20713,3 sendrep: NS002 - send error : Broken pipe
07/18 12:57:32.212 20713,3 Cns_srv_readdir: returns 0
07/18 12:57:32.297 20713,3 sendrep: NS002 - send error : Broken pipe
\end{lstlisting}

Broken pipe error can be written down in more variants - with less send error lines or after arbitrary readdir line.

\subsection{Discussion}

The most important observation that has to be described is: when lcg-del, lcg-cp or lcg-rep commands are invoked from the same ui, the same user and on the same file then there is no possibility to resolve which command has been executed because all commands are written down to the log file by the same lines. The second most important observation is: many lfc commands are written down very similarly and it is very difficult to differentiate between them. 

\section*{LFC analyzer manual}

LFC analyzer is a standard command line application written in C++ language. LFC analyzer allows to present information stored in LFC log files in human readable form. LFC analyzer determines LFC commands that have been invoked by LFC. LFC analyzer keeps several items that are important to analyze log files. These items are: user who has invoked the LFC command, user interface that has been used, file on which the command has been invoked on and the command result.

LFC analyzer allows to present analyzed results in many variants. These variants are selected by standard command line options. LFC analyzer uses two kinds of options: with and without an argument. The only option with an argument which is required is the option -i <input log file>. LFC analyzer counts the number of LFC commands and splits it between successful and unsuccessful into a simple table (see~\ref{out:commandtable}).

\begin{lstlisting}[style=cli, label={out:commandtable}, caption={lfc\_analyzer -i logfile}]
|command      |    failed |      good |
|lfc-ls       |    252453 |     45370 |
|lcg-cr       |         0 |     12580 |
|lcg-rep      |     15943 |        58 |
|lfc-mkdirs   |       313 |        48 |
|lcg-cp       |         ? |      5522 |
|ping db      |         0 |       108 |
|lcg-del      |         ? |      5420 |
|serv errors  |     19798 |         0 |
|utimes       |         0 |        96 |
|lcg-aa       |         ? |      1103 |
|lcg-rm       |         ? |         0 |
lcg-rep failed v sobe zahrnuje lcg-cp failed a lcg-del failed
\end{lstlisting}

Some commands are assigned by character ?. This character can mean one of two things: LFC analyzer was not able to recognize and to analyze the command yet or more than one commands are written down the same to the log file. The second case is for example: commands lcg-rep, lcg-cp, and lcg-del are written down in the same way to the log file. Therefore LFC analyzer informs user about this fact with an informative line below the table (see~\ref{out:commandtable}).

\subsection{Command line options}

LFC analyzer presents analysis output in table form. Number and order of table columns are defined by user options. User can specify options:


\begin{description}
    \item[] -f: Destination file that has been accessed by LFC command.
    \item[] -s: User interface where LFC command has been invoked. 
    \item[] -u: User name that invoked LFC command.
    \item[] -c: LFC command that has been invoked.
    \item[] -r: Result (Success) of invoked LFC command.
    \item[] -t: Time duration and std. deviation for each command.
    \item[] -m: Present command information.
    \item[] -e: Prints time duration of each command, commands are collected to command type and result groups.
Time is presented in "start time | duration" format.
\end{description}

User does not have to specify any options, table will not be presented in this case. For each specified option there is created a column in which the value of the particular item is presented. Columns are ordered according to option order. Example~\ref{out:usercommandtable} shows a table where lfc\_analyzer presents user and LFC commands that they invoked.

\newpage

\begin{center}
\begin{lstlisting}[style=cli, label={out:usercommandtable}, caption={lfc\_analyzer -i logfile -u -c}]
user1   : 6622 |LCG_CP          : 451 |
               |LCG_CR         : 1005 |
               |LCG_DEL          : 21 |
               |LCG_LR          : 133 |
               |LCG_REP        : 3211 |
               |LFC_LS         : 1511 |
               |LFC_MKDIR       : 290 |
user2    : 120 |LCG_UTIME        : 96 |
               |LFC_LS           : 24 |
user3     : 29 |LCG_CP            : 4 |
               |LCG_LR            : 6 |
               |LFC_LS           : 19 |
user3 : 126221 |LCG_AA          : 861 |
               |LCG_CP         : 1033 |
               |LCG_CR         : 8786 |
               |LCG_DEL        : 4177 |
               |LCG_LR         : 8194 |
               |LCG_REP       : 10024 |
               |LFC_LS        : 93083 |
               |LFC_MKDIR        : 61 |
               |SRV_ERR           : 2 |
user5  : 60195 |LCG_AA          : 242 |
               |LCG_CP         : 4030 |
               |LCG_CR         : 2789 |
               |LCG_DEL        : 1222 |
               |LCG_LR        : 19743 |
               |LCG_REP        : 2745 |
               |LFC_LS        : 29400 |
               |LFC_MKDIR        : 10 |
               |SRV_ERR          : 14 |
user6 : 173811 |LCG_CP            : 4 |
               |LCG_REP          : 21 |
               |LFC_LS       : 173786 |
user7  : 19890 |LCG_PINGDB      : 108 |
               |SRV_ERR       : 19782 |
\end{lstlisting}
\end{center}

\newpage

Example~\ref{out:usercommandresulttable} shows a table where lfc\_analyzer presents user, LFC command and result of LFC command.

\begin{center}
\begin{lstlisting}[style=cli, label={out:usercommandresulttable}, caption={lfc\_analyzer -i logfile -u -c -r}]
user1   : 6622 |LCG_CP       : 451 |succeed   : 451 |
               |LCG_CR      : 1005 |succeed  : 1005 |
               |LCG_DEL       : 21 |succeed    : 21 |
               |LCG_LR       : 133 |succeed   : 133 |
               |LCG_REP     :   30 |succeed    : 30 |
               |LFC_LS      : 1511 |succeed   : 188 |
               |                   |failed   : 1323 |
               |LFC_MKDIR    : 290 |failed    : 290 |
user2    : 120 |LCG_UTIME     : 96 |succeed    : 96 |
               |LFC_LS        : 24 |failed     : 24 |
user3     : 29 |LCG_CP         : 4 |succeed     : 4 |
               |LCG_LR         : 6 |succeed     : 6 |
               |LFC_LS        :  2 |succeed     : 2 |
user4 : 126221 |LCG_AA       : 861 |succeed   : 861 |
               |LCG_CP      : 1033 |succeed  : 1033 |
               |LCG_CR      : 8786 |succeed  : 8786 |
               |LCG_DEL     : 4177 |succeed  : 4177 |
               |LCG_LR      : 8194 |succeed  : 8194 |
               |LCG_REP    : 10024 |succeed    : 28 |
               |                   |failed   : 9996 |
               |LFC_LS     : 93083 |succeed : 35162 |
               |                   |failed  : 57921 |
               |LFC_MKDIR     : 61 |succeed    : 42 |
               |                   |failed     : 19 |
               |SRV_ERR        : 2 |failed      : 2 |
user5  : 60195 |LCG_AA       : 242 |succeed   : 242 |
               |LCG_CP      : 4030 |succeed  : 4030 |
               |LCG_CR      : 2789 |succeed  : 2789 |
               |LCG_DEL     : 1222 |succeed  : 1222 |
               |LCG_LR     : 19743 |succeed : 19743 |
               |LCG_REP     : 2745 |failed   : 2745 |
               |LFC_LS     : 29400 |succeed : 10014 |
               |                   |failed  : 19386 |
               |LFC_MKDIR     :  6 |succeed     : 6 |
               |SRV_ERR       : 14 |failed     : 14 |
user6 : 173811 |LCG_CP         : 4 |succeed     : 4 |
               |LCG_REP       : 21 |failed     : 21 |
               |LFC_LS    : 173786 |succeed     : 4 |
               |                   |failed : 173782 |
user7  : 19890 |LCG_PINGDB   : 108 |failed    : 108 |
               |SRV_ERR    : 19782 |failed  : 19782 |
\end{lstlisting}
%\end{tabular}
\end{center}

LFC analyzer also allows to present only selected rows from the result table. User can select each value in result table by options. LFC analyzer provides several options for this purpose.

\begin{description}
    \item[] -d UI : filter by user interface
    \item[] -g USER : filter by user
    \item[] -l FILE : filter by file name
    \item[] -p COMMAND : filter by command name (lfc-ls, lcg-cr, lcg-rep, etc.)
    \item[] -o ( true | false ) : filter by command result
\end{description}

Example~\ref{out:filter1} shows table where user name contains Albert and his failed commands were selected.

\begin{center}
\begin{lstlisting}[style=cli, label={out:filter1}, caption={lfc\_analyzer -i log-20121201 -ucr -g Albert -o false}]
Mr. Albert : 173803 |LCG_REP    : 21 |failed     : 21 |
                    |LFC_LS : 173782 |failed : 173782 |
\end{lstlisting}
\end{center}

%----------------------------------------------

\section*{Analysis result in graphs}

Following graphs were created based on LFC analyzer results. The two analyzed weeks were:

\begin{itemize}
	\item 22.10.2012 - 28.10.2012
	\item 2.9.2013 - 8.9.2013
\end{itemize} 

\begin{figure}[H]
  \centering
    \includegraphics[width=1\textwidth]{figures/commands.png}
\end{figure}

%lfc-ls       
%lcg-cr       
%lcg-rep      
%lfc-mkdir   
%lcg-cp       
%ping db      
%lcg-del      
%serv errors  
%utimes       
%lcg-aa       
%lcg-rm       

\subsection{Results for week 22.10.2012 - 28.10.2012}

\begin{figure}[H]
  \centering
    \includegraphics[width=1\textwidth]{figures/lfc-ls-1.png}
\end{figure}
\begin{figure}[H]
  \centering
    \includegraphics[width=1\textwidth]{figures/lcg-cr-1.png}
\end{figure}
\begin{figure}[H]
  \centering
    \includegraphics[width=1\textwidth]{figures/lcg-rep-1.png}
\end{figure}
\begin{figure}[H]
  \centering
    \includegraphics[width=1\textwidth]{figures/lfc-mkdir-1.png}
\end{figure}
\begin{figure}[H]
  \centering
    \includegraphics[width=1\textwidth]{figures/lcg-del-1.png}
\end{figure}
\begin{figure}[H]
  \centering
    \includegraphics[width=1\textwidth]{figures/serv-errors-1.png}
\end{figure}
\begin{figure}[H]
  \centering
    \includegraphics[width=1\textwidth]{figures/utimes-1.png}
\end{figure}
\begin{figure}[H]
  \centering
    \includegraphics[width=1\textwidth]{figures/lcg-aa-1.png}
\end{figure}

\begin{figure}[H]
  \centering
    \includegraphics[width=1\textwidth]{figures/commands-1.png}
\end{figure}

\subsection{Results for week 2.9.2013 - 8.9.2013}

\begin{figure}[H]
  \centering
    \includegraphics[width=1\textwidth]{figures/lfc-ls-2.jpg}
\end{figure}
\begin{figure}[H]
  \centering
    \includegraphics[width=1\textwidth]{figures/lcg-cr-2.png}
\end{figure}
\begin{figure}[H]
  \centering
    \includegraphics[width=1\textwidth]{figures/lcg-rep-2.png}
\end{figure}
\begin{figure}[H]
  \centering
    \includegraphics[width=1\textwidth]{figures/lfc-mkdir-2.png}
\end{figure}
\begin{figure}[H]
  \centering
    \includegraphics[width=1\textwidth]{figures/ping-db-2.png}
\end{figure}
\begin{figure}[H]
  \centering
    \includegraphics[width=1\textwidth]{figures/lcg-del-2.png}
\end{figure}
\begin{figure}[H]
  \centering
    \includegraphics[width=1\textwidth]{figures/serv-errors-2.png}
\end{figure}
\begin{figure}[H]
  \centering
    \includegraphics[width=1\textwidth]{figures/utimes-2.png}
\end{figure}
\begin{figure}[H]
  \centering
    \includegraphics[width=1\textwidth]{figures/lcg-aa-2.png}
\end{figure}

\begin{figure}[H]
  \centering
    \includegraphics[width=1\textwidth]{figures/commands-2.png}
\end{figure}


\subsection{Week comparison}

\begin{figure}[H]
  \centering
    \includegraphics[width=1\textwidth]{figures/lfc-ls-c.png}
\end{figure}
\begin{figure}[H]
  \centering
    \includegraphics[width=1\textwidth]{figures/lcg-cr-c.png}
\end{figure}
\begin{figure}[H]
  \centering
    \includegraphics[width=1\textwidth]{figures/lcg-rep-c.png}
\end{figure}
\begin{figure}[H]
  \centering
    \includegraphics[width=1\textwidth]{figures/lfc-mkdir-c.png}
\end{figure}
\begin{figure}[H]
  \centering
    \includegraphics[width=1\textwidth]{figures/ping-db-c.png}
\end{figure}
\begin{figure}[H]
  \centering
    \includegraphics[width=1\textwidth]{figures/lcg-del-c.png}
\end{figure}
\begin{figure}[H]
  \centering
    \includegraphics[width=1\textwidth]{figures/serv-errors-c.png}
\end{figure}
\begin{figure}[H]
  \centering
    \includegraphics[width=1\textwidth]{figures/utimes-c.png}
\end{figure}
\begin{figure}[H]
  \centering
    \includegraphics[width=1\textwidth]{figures/lcg-aa-c.png}
\end{figure}

\subsection{LFC analyzer uncertainty}

LFC analyzer is not able to analyze all the items in a LFC log file yet, but it is able to present how many items were recognized and how many were not.

\begin{figure}[H]
  \centering
    \includegraphics[width=1\textwidth]{figures/items-1.png}
\end{figure}

\begin{figure}[H]
  \centering
    \includegraphics[width=1\textwidth]{figures/items-2.png}
\end{figure}


%-----------------------------------------------

\section*{Application documentation}

LFC analyzer is composed from several modules that provide different functionality. These modules process an input from previous modules and provide output for consecutive modules. Input for the first module is a LFC log file. This chain is presented in Figure~\ref{fig:main-seq}. Functionality of each module will be described in the following sections. 

\begin{figure}[H]
	\centering
	\includegraphics[width=1\textwidth]{figures/lfc-main-seq.pdf}
	\caption{The whole analyzing process}
	\label{fig:main-seq}
\end{figure}


\subsection{Method main}

Main method creates and invokes all main modules that are presented in Figure~\ref{fig:main-seq}. The first thing that the method main does is command line arguments parsing. Library getopt has been chosen to parse command line input. The result of parsing (which argument and argument order) is stored to file, site, user, command, resultType variable and to filteredFile, filteredSite, filteredUser, filteredCommand, filteredSuccess variables. These variables influence composition of result module's chain. The next step that method main provides is LFC log file parsing. The method parse from class Parser has been created for this purpose (see~\ref{c:parser}). This method parses the whole log file and returns LogTable object where all parsed items are stored (see~\ref{log:item}). The LogTable object is passed to the method Analyze of Analyzer class. The Analyzer class has been created to recognize and extract all LFC commands that have been written to log files as rows (see~\ref{log:row}). The output of Analyze method is a vector (C++ structure) containing all recognized LFC commands from the input LFC log file. The LFC Analyzer also allows to filter recognized LFC commands. Class Filter has been created for this reason. Command line arguments that require values (value that will be filtered) set up the Filter class. The Filter class filters LFC commands by Filtrate method according to set values. The result that the Filtrate method returns is a vector containing LFC commands for which filtering conditions were met. The next module that processes Filters result is Sorter. Sorter sorts Filter results according to passed command line arguments. For each LFC command property one sorter class object is created. The last module is called Presenter. The presenter module prints filtered and sorted LFC commands to the standard output.

\subsection{Class Parser}
\label{c:parser}

Class parser contains the parse method which parses the whole input LFC log file to a list of Items. The first operation that has to be done is row parsing, therefore parse method parses rows first - extracts log time, log function and thread id. Remain string is stored into another variable. If parse method knows thread id for each row then the method can merge rows with the same thread id. These rows compose one item. A particular parser has been created to parse particular item type. Special inheritance system has been created for this reason. Each parser inherits from class named CommandParser. This class contains default parsing implementation and its descendants can override this parent implementation (see~\ref{fig:commandparser}). 

\begin{figure}[H]
  \centering
    \includegraphics[width=0.8\textwidth]{figures/Lfc-commandParser-class.pdf}
    \caption{Item parser inheritance design}
    \label{fig:commandparser}
\end{figure}

Each descendant of CommandParser overrides (or not if default implementation is sufficient) several parseStateX methods according to number of rows from which is particular Item composed - each parseStateX method parses information from its row (parseState0 parses the first row, parseState1 parses the second row). An instance of class Item is a result of each parsing. Each Item contains time of the first and last Item's row, Item's command, thread id, file path on which was LFC command item invoked, user who invoked LFC Command, User Interface (site) from which has the command been invoked and also Item's command type (see~\ref{fig:item}).

\begin{figure}[H]
  \centering
    \includegraphics[width=1\textwidth]{figures/lfc-analyzer-item-class.pdf}
    \caption{Item class}
    \label{fig:item}
\end{figure}

\subsection{Class Analyzer}
\label{c:analyzer}

The Analyzer class was implemented to recognize LFC commands from parsed Item's. Special recognification design that is similar to the parser design has been implemented to recognize LFC commands. The parsing result is passed to the Analyze method. An abstract parent class State has been created to store information about the currently processed LFC Command (see~\ref{fig:states}). The Analyze method goes through the vector of Items and according to Item's type chooses particular state. The items iterator, start item, and several subitems from the Item's list are passed to the chosen State. State tries to find out the next Item that is part of the particular LFC command. Afterwards, state creates new state and passes to it its new result (start item, item iterator and several items to process). Each state determines next state and creates it. If the state determines that the next item will not follow then it returns the result as a LFC Command. The LFC command is returned to the top of the state stack then next item can start new analyzing process. The LFC analyzer state principle is depth first search principle in general, but with certain condition. The result of analyzing is a vector of LFC Commands that is returned by the Analyze method.

\begin{figure}[H]
  \centering
    \includegraphics[width=1\textwidth]{figures/Lfc-analyzer-state.pdf}
    \caption{State inheritance system}
    \label{fig:states}
\end{figure}

\subsection{Class Filter}
\label{c:filter}

The Filter class is intended to filter values that the user sets by command line options. The user specifies values that each LFC command has to contain by several command line options. The specified values are set by SetSearchedCommand, SetSearchedFileString, SetSearchedSiteString, SetSearchedSuccessValue, SetSearchedUserString setters of Filter class. Method Filtrate starts filtering. If some of set values is not in a LFC Command then the LFC Command is discarded. The vector of filtered LFC Commands is returned by Filtrate method.

\subsection{Class Sorter}
\label{c:sorter}

Set of sorter classes were implemented to create output for user in convenient form. Sorter uses chain of responsibility pattern (see~\ref{fig:sorters}). The abstract Sorter class was implemented as a parent for each Sorter's class. Descendants of parent Sorter are CommandSorter, FileSorter, UserSorter, SucessSorter, and SiteSorter. How these sortes sorts presents list~\ref{enum:sorters}.

\begin{figure}[H]
  \centering
\begin{itemize}
\item CommandSorter: sorts by LFC Command name.
\item FileSorter: sorts by File name on which has the LFC Command been invoked.
\item UserSorter: sorts by User name that LFC Command invoked.
\item SucessSorter: sorts onto two kinds of result: succeed and failed.
\item SiteSorter: sorts according to User Interface name.
\item TimeSorter: sorts according to command time duration.
\item InformationSorter: sorts according to note stored in command.
\end{itemize}
\caption{Sorter classes}
\label{enum:sorters}
\end{figure}

Sorters could be composed together (e.g., CommandSorter can contain FileSorter and it can contain other sorter). This implementation allows to sort result gradually which means that when we want to sort LFC commands by success first and afterward to sort the result according to file name, we can compose Sorter's constructor according to example~\ref{sorters}. The SuccessSorter splits commands onto two groups - failed and successful command. Afterward, the FileSorter is called to sort each group of SuccessSorter's result). This principle is the same for each Sorter - the previous Sorter sorts values to several groups and the next Sorter is called to sort each group separately.

\begin{center}
\begin{lstlisting}[style=cli, label={sorters}, caption={Composed sorters}]
Sorter * sorter = new FileSorter(new SuccessSorter());
\end{lstlisting}
\end{center}

\begin{figure}[H]
  \centering
    \includegraphics[width=1\textwidth]{figures/Lfc-Sorter-class-1.pdf}\
    \caption{Sorter class diagram}
    \label{fig:sorters}
\end{figure}

\subsection{Class Presenter}
\label{c:presenter}

The Presenter class prints out the result in table form. The main method that presents the result is \textit{Print(vector<LFCCommad *> commands, int file, int user, int site, int command, int resultType)}. The first argument is sorted vector of commands that will be presented and remaining arguments are variables that contain order to order table columns. Table columns are presented and ordered according to command line parameters selected by the user. The most important thing about the result table is: the result table prints only first value if the next values are the same in one column. Therefore, user can easily browse trough LFC commands and recognize their values.

%-----------------------------------------------

\section*{Installation}

\begin{verbatim}
    git clone https://github.com/radekludacka/lfc-analyzer.git
    cd lfc-analyzer
    sh install.sh
    ./bin/lfc-analyzer -i <logfile>
    # or ./bin/lfc-analyzer -i <logfile> -ur -o false etc.
\end{verbatim}

\section*{Post analysis}

LFC analyzer allows to create plots for each lfc command and its result (Successful, Failed) combination where x-axis contains 
start times of each command and y-axis contains their durations. We have to launch lfc-analyzer with -cet options and redirect
to an output file. The output file is input for plot-creator.py script. The plot-creator.py creates plot for each
command-result combination to current directory. The invocation sequence can be following:

\begin{verbatim}
    ./bin/lfc-analyzer -i <logfile> -cet > result.txt
    python plot-creator.py result.txt
\end{verbatim}

Plots similarly Figure~\ref{img:plot-example} will be created.

\begin{figure}[H]
  \centering
    \includegraphics[width=1\textwidth]{figures/plot-example.png}
    \caption{Plot presents durations of successful lcg-aa command - output of plot-creator.py script}
    \label{img:plot-example}
\end{figure}


%----------------------------------------------------------------------------------------
%	BIBLIOGRAPHY
%----------------------------------------------------------------------------------------

\bibliographystyle{unsrt}

\bibliography{sample}

%----------------------------------------------------------------------------------------

\end{document}
